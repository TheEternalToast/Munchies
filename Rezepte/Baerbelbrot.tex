\section{Bärbelbrot}
\begin{tikzpicture}[remember picture,overlay]
    \node[anchor=east,yshift=-4.5cm,inner sep=0pt] at (current page text area.east|-0,3cm) {\includegraphics[height=6cm]{res/Bärbelbrot.png}};
\end{tikzpicture}
\subsection*{Vorteig}\label{vorteig}
\begin{longtable}{rlL{15.9cm}}
	330ml				&	Wasser				&	auf \cel{40} erwärmen und mit	\\
	$1\frac{1}{2}$ EL	&	Anstellgut			&	und	\\
	330g				&	Roggenvollkornmehl	&	vermischen	\\
\end{longtable}

Den Vorteig abends zubereiten und \textbf{über Nacht} bei Zimmertemperatur (ca. \cel{20}) gehen lassen.\\
Dann $1\frac{1}{2}$ EL abnehmen und als Anstellgut in einem Schraubglas aufbewahren.

\subsection*{Hauptteig}
\begin{longtable}{rlL{16cm}}
	$\frac{3}{4}$ EL	&	Salz				&	in	\\
	330ml				&	Wasser				&	auflösen und auf \cel{40} erwärmen.	\\
	1 EL				&	Kümmel				&	mit	\\
	1 TL				&	Koriander			&	mörsern.	\\
	1 EL				&	Sesam				&	,	\\
	1 EL				&	Sonnenblumenkerne	&	und	\\
	1 EL				&	Leinsamen			&	mit Koriander-Kümmel Gemisch und	\\
	360g				&	Weizenvollkornmehl	&	vermischen und mit Salzwasser und	\\
						&	\nameref{vorteig}	&	zu einem Teig verkneten.	\\
\end{longtable}

Eine passende Kastenform einfetten und mit Sesam bestreuen, damit sich das Brot später besser herauslösen lässt.\\
Hauptteig in die Kastenform füllen und mit feuchten Fingern glattstreichen.
Dann Sesam andrücken und für \std{3 bis 6} gehen lassen.\\
Anschließend bei \cel{210} bis \cel{230} für \std{1} backen.
Danach sofort stürzen und abkühlen lassen.