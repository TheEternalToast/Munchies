% allgem. Dokumentenformat
\documentclass[12pt,landscape,oneside]{book}


% deutsche Silbentrennung
\usepackage[ngerman]{babel}

% Umlaute unter UTF8 nutzen

% Zeichenencoding
\usepackage[utf8]{inputenc}
\usepackage{lmodern}
\usepackage{fix-cm}

% mehrseitige Tabellen ermöglichen
\usepackage{longtable}

% Packet für Seitenrandabstände und Einstellung für Seitenränder
\usepackage[left=3cm, right=3cm, top=2.25cm, bottom=2.25cm]{geometry}

% neue Kopfzeilen mit fancypaket
\usepackage{fancyhdr} %Paket laden
\pagestyle{fancy} %eigener Seitenstil
\fancyhf{} %alle Kopf- und Fußzeilenfelder bereinigen
\fancyhead[L]{\nouppercase{\leftmark}} %Kopfzeile links
\fancyhead[C]{} %zentrierte Kopfzeile
\fancyhead[R]{\thepage} %Kopfzeile rechts
\renewcommand{\headrulewidth}{0.4pt} %obere Trennlinie
\setlength{\headheight}{15pt}

%Fußnoten
\usepackage{footnote}
\usepackage[perpage,symbol,stable]{footmisc}
\makeatletter
\newcommand\footnoteref[1]{\protected@xdef\@thefnmark{\ref{#1}}\@footnotemark}
\makeatother
\makesavenoteenv{tabular}
\makesavenoteenv{table}


% für Tabellen
\usepackage{array}
\usepackage{booktabs}
\usepackage{tabularx}
\newcolumntype{L}[1]{>{\raggedright\arraybackslash}p{#1}}

% Paket für Zeilenabstand
\usepackage{setspace}

% Verlinkungen innerhalb des Dokuments
\usepackage[hidelinks]{hyperref}
\hypersetup{allcolors=black,linktocpage,linktoc=all}
\newcommand{\recipe}[1]{%
    \hyperref[sec:#1]{\nameref{sec:#1} (S. \pageref{sec:#1})}
}

% Nummerirung der Rezepte abschalten
\setcounter{secnumdepth}{-2}

% Ungetestete Rezepte markieren
\usepackage[dvipsnames]{xcolor}
\newcommand{\unverified}[1][UNGEPRÜFTES REZEPT -- Anweisungen ggf. anpassen]{%
    \centering%
    \textcolor{yellow}{#1}%
    \flushleft%
}
\newcommand{\wip}[1][WORK IN PROGRESS]{%
    \centering%
    \textcolor{red}{#1}%
    \flushleft%
}

% weitere Pakete
% Grafiken aus PNG Dateien einbinden
\usepackage{graphicx}
\usepackage{tikzpagenodes}

% Titel- und Rückseite
\usepackage{pdfpages}

% Hintergrundbilder
\usepackage{transparent}
\usepackage{eso-pic}

% Einheiten
\usepackage{textcomp}
\newcommand{\cel}[1]{\textbf{#1\textdegree{}C}}
\newcommand{\mins}[1]{\textbf{#1 min}}
\newcommand{\std}[1]{\textbf{#1 Std.}}

% Schriftart
\renewcommand{\familydefault}{\sfdefault}
\date{}