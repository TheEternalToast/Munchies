\section{Stapfelrisotto}
\begin{longtable}{rlL{17,06cm}}
    3       &   Zwiebeln        &   fein und \\
    100g    &   Steinpilze      &   grob schneiden und in  \\
    etwas   &   Butter\footnote{gerne auch vegan z.B. \href{https://www.bio123.de/produkt/naturli/naturli-organic-vegan-block-200g}{Naturli vegan block}
                            oder \href{https://www.alsan.de/alsan-bio/}{Alsan}}
                                &   anschmoren.  \\
    2       &   Äpfel           &   in Stücke schneiden und mit \\
    200g    &   Risottoreis     &   zusammen in die Pfanne geben.
                                    Etwas braten lassen, dann mit   \\
    150ml   &   Weißwein        &   ablöschen und kochen lassen.
                                    Dabei unter ständigem Rühren nach und nach  \\
    500ml   &   Gemüsebrühe     &   zugeben.    \\
    75g     &   Parmesan        &   reiben und unterrühren, wenn die Flüssigkeit fast vollständig verkocht und aufgesaugt wurde.
                                    Mit \\
            &   Salz \& Pfeffer &   abschmecken.\\
\end{longtable}

Das Gericht lässt sich gut verfeinern, indem man z.B. \textbf{Walnüsse} röstet und über das fertige Risotto streut.\\
Als Ersatz für die Steinpilze (die ja nicht immer leicht zu finden und im Laden meistens teuer sind) eignen sich Pfifferlinge.
Champignons sind auch möglich, verlieren beim Garen allerdings deutlich mehr Flüssigkeit und sollten daher erst mit dem Reis in die Pfanne gegeben werden.
Da sie weniger Geschmacksintensiv sind, empfiehlt es sich auch etwas mehr zu verwenden.