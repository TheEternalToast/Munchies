\section{Veganer Käsekuchen}
\subsection*{Vorbereitung}
\begin{longtable}{rlL{17.6cm}}
	150g	&	Cahewnüsse	&	mit kochendem Wasser bedecken, 1 Stunde ziehen lassen und dann gründlich abtropfen lassen.
\end{longtable}

Ofen auf \cel{180} vorheizen.
\subsection*{Der Boden}
\begin{longtable}{rlL{16.8cm}}
	200g	&	Spekulatiuskekse	&	(oder vergleichbares) fein zerbröseln. \\
	100g	&	Margarine\footnote{\label{foot:margarine}z.B. \href{https://www.bio123.de/produkt/naturli/naturli-organic-vegan-block-200g}{Naturli vegan block}
                            oder \href{https://www.alsan.de/alsan-bio/}{Alsan}}
									&	schmelzen und mit zerbröselten Keksen mischen und am Boden der Springform platt drücken. \\
\end{longtable}

Bis zu \mins{20} bei \cel{180} backen. (Dieser Schritt ist nicht zwingend nötig, sorgt aber für einen etwas stabileren Boden.)\\
Danach ein Blech mit Wasser füllen, in den Ofen stellen und Temperatur auf \cel{160} reduzieren.
	
\subsection*{Die Füllung}
\begin{longtable}{rlL{15.49cm}}
	1		&	Orange				&	auspressen und Zesten reißen und von\\
	2 Dosen	&	Kokosmilch			&	den Rahm abschöpfen. Den Orangensaft und die Zesten und den Kokosrahm mit den vorbereiteten Cashews,\\
	300g	&	veganem Frischkäse	&	,\\
	200ml	&	Ahornsirup			&	oder\\
	150g	&	Zucker				&	,\\
	20g		&	Margarine\footnoteref{foot:margarine}
									&	und\\
	1 Prise	&	Salz				&	in einem Mixer zu einer cremigen, gleichmäßigen Masse zerhäckseln.
										einen Teil davon abschöpfen und mit\\
	1 1/2 EL	&	Speisestärke	&	verrühren. Diese Masse gründlich mit dem Rest vermischen und
										vorsichtig auf den Boden gießen. \\
\end{longtable}

Glatt ruckeln bis keine neuen Bläschen mehr an die Oberfläche treten und diese dann vorsichtig abschöpfen.
Ca. \mins{10} backen dann Ofen auf \cel{125} runterdrehen und \mins{85} backen (Wenn sich eine Haut bildet, ist er fertig.
Der Stechtest funktioniert bei diesem Kuchen \textbf{nicht}.). Den Kuchen langsam abkühlen lassen.