\section{Ramen}\label{sec:ramen}
\subsection*{Suppenbasis}\label{subsec:ramen_soup}
\begin{longtable}{rlL{16.2cm}}
	2 Zehen	&	Knoblauch			&	und	\\
	1,5cm	&	Ingwer				&	fein schneiden.	\\
	1		&	Frühlingszwiebel	&	in Ringe schneiden und dabei den weißen vom grünen Teil trennen.	\\
	2 EL	&	Sesamöl				&	in einem Topf erhitzen und	Knoblauch, Ingwer und den weißen Teil der Frühlingszwiebeln bei schwacher bis mittlerer Hitze anschwitzen.	\\
	2 EL	&	Doubanjiang\footnote{fermentierte Bohnenpaste (mit oder ohne Chili)}	&	und \\
	2 EL	&	Miso				&	(weiß oder rot je nach Geschmack) zugeben und konstant rühren, sodass sie nicht anbrennen.	\\
	1 EL	&	Sake				&	hinzugeben. (Das hilft dabei, festgebackene Stückchen zu lösen.)	\\
	1 EL	&	weißer Sesam		&	mörsern und zusammen mit	\\
	2 EL	&	heller Sojasoße		&	hinzugeben.	\\
	240ml	&	Sojamilch			&	nach und nach (!) hinzugeben.
										Dann	\\
	120ml	&	\recipe{dashi}		&	hinzufügen und mit	\\
			&	weißem Pfeffer		&	und	\\
			&	Salz				&	abschmecken.	\\
\end{longtable}
Die Suppenbasis darf \textbf{nicht} kochen, sonst fällt das Miso aus!
Anschließend die Basis nach belieben mit Wasser aufgießen.
Für ein besonders indviduelles Geschmackserlebnis kann man das auch direkt in der Schüssel machen.
\subsection*{Nudeln}\label{subsec:ramen_noodles}
\begin{longtable}{rlL{16.7cm}}
	250g	&	Mehl (Typ 405)	&	mit \\
	15g		&	Gluten			&	vermischen und eine Mulde hineindrücken. \\
	3g		&	Salz			&	und \\
	2g		&	Na$_2$CO$_3$	&	in \\
	100ml	&	warmen Wasser	&	auflösen und in die Mulde geben.
									Von innen nach außen einrühren, sodass das Mehl nach und nach aufgenommen wird. \\
\end{longtable}

Sobald der Teig nicht mehr klebrig ist, das restliche Mehl nach und nach in den Teig einfalten bis ein glatter Teig entsteht.
Den Teig für ca \mins{30} im Kühlschrank ruhen lassen.

In einer Nudelmaschine ausrollen und dabei bei Bedarf wieder falten, bis der Teig dünn genug ist, um zu Nudeln mit quadratischem Querschnitt zu schneiden.

Ungesalzenes Wasser kochen und die Nudeln für \textbf{30 Sekunden} kochen.
\subsection*{Toppings}\label{subsec:ramen_toppings}
Offensichtlich nach belieben (das ist ja gerade das schöne an Ramen).
Aber geeignet sind (unter anderem!):
\begin{itemize}
	\item der (schon geschnittene) grüne Teil der Frühlingszwiebel
	\item (Ramen-)Eier (Rezept t.b.a.)
	\item (am besten frische) Bohnensprösslinge (Soja oder Mungo)
	\item Mais
	\item Kimchi
	\item Röstzwiebeln (ganz zum Schluss oben drauf, weichen sofort durch)
\end{itemize}
\subsection*{Servieren}
\nameref{subsec:ramen_noodles} zubereiten, dann sofort mit \nameref{subsec:ramen_soup} auffüllen und mit \nameref{subsec:ramen_toppings} garnieren.
Schlürfen nicht vergessen!